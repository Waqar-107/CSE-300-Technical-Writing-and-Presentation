\documentclass[14pt]{article}

\usepackage{amsmath}
\usepackage{tikz}

%opening
\title{Pythagorean Theorem}
\author{1505107}


\begin{document}

\maketitle

\definecolor{lightViolet}{RGB}{164,117,169}
\definecolor{skyBlue}{RGB}{215,235,242}

\section{Introduction}
In this document, we present the very famous theorem in mathematics: \textit{ Pythagorean
theorem}, which is stated as follows.\\
\\
\textbf{Theorem 1.1 (Pythagorean theorem)}\textit{The square of the hypotenuse (the
	side opposite the right angle) is equal to the sum of the squares of the other two
	sides.}\\
\\
Numerous mathematicians proposed various proofs to the theorem. The
theorem was long known even before the time of Pythagoras. Pythagoras was
the first to provide with a sound proof. The proof that Pythagoras gave was
by \textit{rearrangement}. Even the great Albert Einstein also proved the theorem
without rearrangement, rather by using dissection. Figure 1 shows the visual
representation of the theorem.\\\\

\begin{figure}[h]
	\centering
	
	\begin{tikzpicture}[scale=0.75]
	
	\draw[fill=skyBlue,skyBlue] (0,-0.5) rectangle (10,7); 
	
	%----------------------------------------------------------
	\draw[fill=cyan,cyan] (2,0) rectangle (4.5,2.5);
	\node[color=violet] at (3.25,1.25) {$c^2$};
	
	\node[color=violet] at (5,1) {=};
	
	\draw[fill=lightViolet,lightViolet] (5.5,0.50) rectangle (6.5,1.50);
	\node[color=violet] at (6,1) {$a^2$};
	
	\node[color=violet] at (7,1) {+};
	
	\draw[fill=pink,pink] (7.5,0.25) rectangle (9.5,2.25);
	\node[color=violet] at (8.5,1.25) {$b^2$};
	%----------------------------------------------------------
	
	%----------------------------------------------------------
	\draw[fill=lightViolet,lightViolet] (4.5,3.5) rectangle (5.5,4.5);
	\draw[color=red] (4.5,4.5) -- (5.5,4.5);
	\node[color=violet] at (5,4) {$a$};
	
	\draw[fill=pink,pink] (5.5,4.5) rectangle (7.5,6.5);
	\draw[color=red] (5.5,4.5) -- (5.5,6.5);
	\node[color=violet] at (6.5,5.5) {$b$};
	
	\draw[color=red] (4.5,4.5) -- (5.5,6.5);
	
	%90-degree will be drawn using two lines
	\draw[color=blue] (5.25,4.75) -- (5.5,4.75);
	\draw[color=blue] (5.25,4.5) -- (5.25,4.75);
	%----------------------------------------------------------
	
	\end{tikzpicture}
	
	\caption{Visual representation of the famous Pythagorean theorem.}
	\label{fig:1}
\end{figure}

\section{Trigonometric Forms}
Lots of other forms of the same theorem exist. The most useful, perhaps, are
expressed in trigonometric terms, as follows:\\

\begin{equation}
	sin^2\theta + cos^2\theta = 1
	\label{eqn:e1}
\end{equation}

\begin{equation}
	sec^2\theta - tan^2\theta = 1 
	\label{eqn:e2}
\end{equation}

\begin{equation}
	cosec^2\theta - cot^2\theta = 1
	\label{eqn:e3}
\end{equation}

\subsection{Representing the First}
Taking \ref{eqn:e1}, we can show them as shown in Figure2. When we take a point at
unit distance from the origin, the $y$ and $x$ co-ordinates become $sin\theta$ and $cos\theta$
respectively. Therefore, sum of the squares of the two becomes equal to the
square of the unit distance, which of course, is 1.

\end{document}
