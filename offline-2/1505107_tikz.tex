\documentclass[14pt]{article}

\usepackage{amsmath}
\usepackage{tikz}

%opening
\title{Pythagorean Theorem}
\author{1505107}


\begin{document}

\maketitle

\definecolor{lightViolet}{RGB}{164,117,169}
\definecolor{skyBlue}{RGB}{215,235,242}
\definecolor{darkBlue}{RGB}{37,45,141}

\section{Introduction}
In this document, we present the very famous theorem in mathematics: \textit{ Pythagorean
theorem}, which is stated as follows.\\
\\
\textbf{Theorem 1.1 (Pythagorean theorem)}\textit{The square of the hypotenuse (the
	side opposite the right angle) is equal to the sum of the squares of the other two
	sides.}\\
\\
Numerous mathematicians proposed various proofs to the theorem. The
theorem was long known even before the time of Pythagoras. Pythagoras was
the first to provide with a sound proof. The proof that Pythagoras gave was
by \textit{rearrangement}. Even the great Albert Einstein also proved the theorem
without rearrangement, rather by using dissection. Figure 1 shows the visual
representation of the theorem.\\\\

\begin{figure}[h]
	\centering
	
	\begin{tikzpicture}[scale=0.65]
	
	\draw[fill=skyBlue,skyBlue] (0,-0.5) rectangle (10,8); 
	
	%----------------------------------------------------------
	\draw[fill=cyan,cyan] (2,0) rectangle (4.5,2.5);
	\node[color=violet] at (3.25,1.25) {$c^2$};
	
	\node[color=violet] at (5,1) {=};
	
	\draw[fill=lightViolet,lightViolet] (5.5,0.50) rectangle (6.5,1.50);
	\node[color=violet] at (6,1) {$a^2$};
	
	\node[color=violet] at (7,1) {+};
	
	\draw[fill=pink,pink] (7.5,0.25) rectangle (9.5,2.25);
	\node[color=violet] at (8.5,1.25) {$b^2$};
	%----------------------------------------------------------
	
	%----------------------------------------------------------
	\draw[fill=lightViolet,lightViolet] (4.5,3.5) rectangle (5.5,4.5);
	\draw[color=red] (4.5,4.5) -- (5.5,4.5);
	\node[color=violet] at (5,4) {$a$};
	
	\draw[fill=pink,pink] (5.5,4.5) rectangle (7.5,6.5);
	\draw[color=red] (5.5,4.5) -- (5.5,6.5);
	\node[color=violet] at (6.5,5.5) {$b$};
	
	%rotated a, the difficult part
	\draw[fill=cyan,cyan,shift={(7.55 cm,0.5 cm)},rotate=64] (2.264,4.5) rectangle (4.5,6.74);
	\node[color=violet] at (4.5,5.5) {$a$};
	
	
	\draw[color=red] (4.5,4.5) -- (5.5,6.5);
	
	%90-degree will be drawn using two lines
	\draw[color=blue] (5.25,4.75) -- (5.5,4.75);
	\draw[color=blue] (5.25,4.5) -- (5.25,4.75);
	%----------------------------------------------------------
	
	\end{tikzpicture}
	
	\caption{Visual representation of the famous Pythagorean theorem.}
	\label{fig:1}
\end{figure}


\begin{figure}[h]
	\centering
	\begin{tikzpicture}
		%\draw[help lines] (-5,-5) grid (5,5);
	
		%other lines
		\draw[color=cyan,line width=1] (0,2) -- (3.5,2);
		\draw[color=black,line width=1] (3.5,2) -- (5,2);
		\draw[color=black,line width=1] (3.5,2) -- (3.5,3.5);
		\draw[color=cyan,line width=1] (3.5,0) -- (3.5,2);
		\draw[color=red,line width=1] (0,0) -- (3.5,2);
		
		%arrows
		%sin()
		\draw[color=darkBlue,ultra thick,<->] (4.5,0) -- (4.5,2);
		\node[color=darkBlue,scale=2] at (5.5,1) {$sin\theta$}; 
		
		%cos()
		\draw[color=darkBlue,ultra thick,<->] (0,2.5) -- (3.5,2.5);
		\node[color=darkBlue,scale=2] at (1.5,3) {$cos\theta$}; 
		
		%theta
		\node[color=cyan,scale=2] at (2.5,0.5) {$\theta$};
		\draw[color=cyan,thick]  (1.85,0) to[out=45,in=0] (1.5,0.85);
		
		%r=1 label;
		\node[color=red,scale=2,rotate=25] at (1,1) {$r=1$};
		
		%axis and circle
		\draw[color=darkBlue,line width=1.5] (-5,0) -- (5,0);
		\draw[color=darkBlue,line width=1.5] (0,-5) -- (0,5);
		\draw[color=darkBlue,ultra thick] (0,0) circle (4cm);
		
		 
	\end{tikzpicture} 
	\caption{Alternate representation of Pythagorean theorem.}
	\label{fig:2}
\end{figure}

\section{Trigonometric Forms}
Lots of other forms of the same theorem exist. The most useful, perhaps, are
expressed in trigonometric terms, as follows:\\

\begin{equation}
	sin^2\theta + cos^2\theta = 1
	\label{eqn:e1}
\end{equation}

\begin{equation}
	sec^2\theta - tan^2\theta = 1 
	\label{eqn:e2}
\end{equation}

\begin{equation}
	cosec^2\theta - cot^2\theta = 1
	\label{eqn:e3}
\end{equation}

\subsection{Representing the First}
Taking \ref{eqn:e1}, we can show them as shown in \ref{fig:2}. When we take a point at
unit distance from the origin, the $y$ and $x$ co-ordinates become $sin\theta$ and $cos\theta$
respectively. Therefore, sum of the squares of the two becomes equal to the
square of the unit distance, which of course, is 1.

\end{document}

%https://tex.stackexchange.com/questions/49169/translate-and-rotate-an-object-in-tikz-2d/49224#49224